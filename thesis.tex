\documentclass[12pt,a4paper,openright]{book} % oneside

%   PACKAGES
\usepackage{lmodern}         % font package.
\usepackage[T1]{fontenc}     % define T1 charset for out files.
\usepackage[italian]{babel}  % italian latex typo conventions.
\usepackage[utf8]{inputenc}  % italian symbols.
\usepackage{amsmath}         % math features.
\usepackage{amsthm}          % math theorems.
\usepackage{amssymb}         % math symbols.
\usepackage{listings}        % embed programming language in latex.
\usepackage{stmaryrd}        % symbols for theoretical computer science.
\usepackage{hhline}          % better horizontal lines in tabulars and arrays.
%%\usepackage{vmargin}         % various page dimensions.
\usepackage{hyperref}        % hypertext support.
\usepackage{makeidx}         % for creating indexes.
\usepackage{nicefrac}        % inline fractions.
\usepackage{marginnote}      % notes in the margin, even where \marginpar fails.
\usepackage{xr}              % references to other latex documents.
\usepackage{subfiles}        % multifile support.
\usepackage{geometry}        % interface for document dimension.
\usepackage{graphicx}        % enhanced support for graphics.
\usepackage{fancyhdr}        % extensive control of page headers and footers.
\usepackage{lipsum}          % generate dummy text.
\usepackage[backend=biber,style=numeric]{biblatex}  % bib management. %bibtex
\usepackage{minitoc}         % table of contents per chapter.
\usepackage{listings}        % code sections, http://ctan.org/pkg/listings
%\usepackage{showframe}      % shows page frames.

%   CONFIGS
\hypersetup{colorlinks=true, urlcolor=blue, linkcolor=blue}
\dominitoc
\lstset{
  basicstyle=\ttfamily,
  mathescape
}

%   RESOURCES
\subfile{prooftree}
\input{macros.tex}
\addbibresource{biblio.bib}


% **********************************************
% **********************************************
\begin{document}

%%% First page
\begin{titlepage}
    \begin{center}       
        \includegraphics[width=0.4\textwidth]{img/logo_unipr.png}

		\vspace{0.5cm}
		
		% *** university details
		\Large
        Dipartimento di Scienze Matematiche\\
        Fisiche ed Informatiche

        \vspace{0.5cm}
        
        \Large
        Corso di laurea in Informatica 
		
		\vspace{1.5cm} 
 
 		% *** title
        \Huge
        \textbf{Some very long title like this and that and this and that again}
 
 		% *** subtitle
        \vspace{1cm}
        \LARGE
        A very very long subtitle: more than the main title, maybe
 
        \vspace{1.5cm}
 
 		% *** authors
		\large Candidato: \Large\textbf{Luca Parolari}\\
		\large Relatore: \Large\textbf{Gianfranco Rossi}
	
		%\begin{minipage}{0.4\textwidth}
		%	\begin{flushleft}
		%		\large
		%		\textit{Author}\\
		%		B.J. \textsc{Blazkowicz} % Your name
		%	\end{flushleft}
		%\end{minipage}
		%
		%\begin{minipage}{0.4\textwidth}
		%	\begin{flushright}
		%		\large
		%		\textit{Supervisor}\\
		%		Dr. Caroline \textsc{Becker} % Supervisor's name
		%	\end{flushright}
		%\end{minipage}
		
        \vfill
 		
 		\large
        Tesi di Laurea in Informatica\\
        Anno Accademico 2018-2019
 
    \end{center}
\end{titlepage}

%%% Dedication
%\begin{dedication}
%test
%\end{dedication}

%%% TOC
\tableofcontents

%\lhead[\fancyplain{}{}]{\fancyplain{}{\leftmark}} \chead{}
%\rhead{\thepage} \lfoot{} \cfoot{} \rfoot{}


%%% *******************************************************
%%% Introduzione
\chapter*{Introduzione}

\lipsum[1-4]


%%% *******************************************************
% Linguaggi a vincoli basati su insiemi e relazioni bianrie
\chapter{Linguaggi a vincoli basati su insiemi e relazioni binarie}
\label{ch:sets-binrel-based-constraint-languages}

\minitoc

In questo capitolo vengono presentati e contestualizzati due linguaggi orientati alla risoluzione di formule insiemistiche: $\clpset$ e $\lbr$. Entrambi i linguaggi seguono il paradigma della Logic Programming e, più precisamente essendo basati su vincoli, della Constraint Logic Programming. Nelle sezioni \ref{sec:lp} e \ref{sec:clp} sono descritti rispettivamente i due paradigmi in forma molto discorsiva ed introduttiva.

Nella sezione\ref{sec:clpset} si da una definizione di $\clpset$, un linguaggio a vincoli basato su insiemi. 
Nel capitolo XXX è descritta l'implementazione di questo linguaggio in Picat, obiettivo di questo lavoro di tesi.

Nella sezione \ref{sec:lbr} è invece definito $\lbr$, un linguaggio che estende ed include $\clpset$, interamente basato su vincoli su relazioni binarie.

\section{Logic Programming}
\label{sec:lp}

La programmazione logica (o Logic Programming, abbreviato \emph{LP}) è un paradigma di programmazione dove gli \emph{statements} del programma rappresentano fatti o regole, espressi tramite qualche logica formale, di un problema. La logica viene utilizzata come meccanismo formale per analizzare le inferenze in termini di operazioni su espressioni simboliche, dedurre conseguenze da un insieme di premesse, studiare la verità (o falsità) di un insieme di proposizioni, data la verità (o falsità) di altre proposizioni e dimostrare la validità di una teoria.

Nel campo dell'informatica la logica trova molteplici applicazioni in ambiti per lo più di intelligenza artificiale. La logica, sotto opportune ipotesi, può diventare un vero e proprio linguaggio di programmazione, prendendo il nome di Logic Programming (per approfondire i concetti di programmazione logica si veda \cite{ConsoleL97}).

Spesso i linguaggi di programmazione di questo tipo adottano un approccio dichiarativo, descrivendo solamente la forma della soluzione e non come ottenerla. \`E il caso di linguaggi come il Datalog. Per i linguaggi come il Prolog invece l'approccio può essere ibrido, intersecando il paradigma dichiarativo, imperativo e tal volta anche procedurale.

La logica classica si può suddividere in due classi principali: la \emph{logica proposizionale} e la \emph{logica dei predicati}. Entrambe permettono di esprimere proposizioni (frasi) e relazioni tra posizioni. La principale differenza tra le due è l'espressività. Ciò che può essere espresso nella logica dei predicati può essere espresso nella logica proposizionale, ma non viceversa. La logica dei predicati consente di utilizzare \textbf{variabili} e quantificazioni su di esse, mentre nella logica proposizionale non è possibile.

\paragraph{Logica dei predicati}
Volendo dare una definizione un po' più formale per la logica dei predicati 

\section{Constraint Logic Programming}
\label{sec:clp}
aaa

\subsection{Constraint Satisfaction Problem}
\label{sec:csp}
Un problema di soddisfacimento di vincoli è composto da tre componenti:
\begin{itemize}
\item un insieme $X$ di variabili ${X_1, \ldots, X_n}$;
\item un insieme $D$ di domini ${D_1, \ldots, D_n}$, uno per variabile;
\item un insieme $C$ di vincoli che specificano le combinazioni di valori possibili per le variabili.
\end{itemize}

Ogni dominio $D_i$ specifica un insieme di valori possibili per la variabile $X_i$.

\section{Il linguaggio $\clpset$}
\label{sec:clpset}
$\clpset$, definito in \cite{clpset2000} è un linguaggio logico per la gestione di vincoli su insiemi. Da un punto di vista più generale $\clpset$ è un istanza dei CLP sopra descritti, dove il dominio di applicazione dei vincoli è quello insiemistico. In questo contesto gli insiemi sono visti come un tipo di dato primitivo del linguaggio, detti termini della logica del primo ordine. I predicati predefiniti invece sono visti come vincoli predefiniti del linguaggio, gestiti con procedure di risoluzione di vincoli. 

La classe di insiemi considerata vuole essere molto generale: gli insiemi possono essere annidati e/o parzialmente specificati. Gli insiemi parzialmente specificati possono contenere variabili e altri elementi "nonground".

Uno degli obiettivi primari di $\clpset$ è di essere molto flessibile e fornire forme generali per la manipolazione di insiemi e relative operazioni, in quanto la nozione di \emph{set} è una componente comune nella progettazione di programmi, ma sono pochi i linguaggi che forniscono gli insiemi come struttura dati elementare.
Sempre in \cite{clpset2000} vengono menzionate alcune eccezioni di linguaggi che si basano su insiemi: SETL, B ed il linguaggio Z, utilizzato per descrivere la specifica formale di programmi. Qualche eccezione esiste anche nel campo dei \emph{database deduttivi} e, più recentemente, anche come \emph{general purpose programming language}. Ad ogni modo, questi linguaggi impongono dei limiti sul tipo di insiemi espressibili o sulle capacità computazionali degli stessi. Ad esempio, in molti ambiti, si richiede che gli insiemi siano totalmente specificati: non sono ammesse \emph{variabili} libere.

Si noti che in $\clpset$ il costo computazione non è un requisito necessario. Molte operazioni possono risultare anche molto costose, ma dal punto di vista implementativo si preferisce implementare la risoluzione al problema in modo dichiarativo e più intuitivo possibile.

\section{Il linguaggio $\lbr$}
\label{sec:lbr}


%%% *******************************************************
%%% Picat
\chapter{Picat}

\minitoc

In questo capitolo viene presentato Picat, un linguaggio di programmazione multiparadigma basato sulla logica. Picat è per certi versi molto simile al Prolog, ma dispone di molte funzionalità che modificano radicalmente l'utilizzo comune sia di un linguaggio imperativo che logico. 

In questo capitolo se ne da una descrizione sommaria delle funzionalità e delle principali caratteristiche, fornendo esempi esplicativi. Non essendo questa una traduzione della guida utente per Picat, si invita il lettore ad approfondire i numerosi dettagli tralasciati per ovvi motivi in questa presentazione sul manuale \cite{PicatGuide}. Si avvisa inoltre che al tempo in cui si sta scrivendo, la guida di Picat è in versione 2.6.

\section{Features}

Picat incorpora le principali funzionalità dei linguaggi logici, funzionali e di scripting. Le caratteristiche di Picat sono riassunte dalle lettere del suo nome:
\begin{itemize}
\item \textbf{P}attern-matching. Un \emph{predicato} definisce una relazione e può avere zero, una o più soluzioni. Picat è un linguaggio basato su regole. I predicati e le funzioni sono definiti tramite regole di pattern-matching.
\item \textbf{I}ntuitive. Picat fornisce comandi come l'assegnamento ed i loop, che vengono usati nei comuni linguaggi imperativi, rendendolo di fatto più affine per la scrittura di programmi comuni.
\item \textbf{C}onstraints. Picat supporta la programmazione con vincoli. Picat nasce con quattro moduli per la risoluzione di vincoli: \verb|cp|, \verb|sat|, \verb|smt| e \verb|mip|.
\item \textbf{A}ctors. Viene fornito il supporto a chiamate event-driven tramite \emph{action rules}, necessarie per descrivere il comportamento degli attori event-driven.
\item \textbf{T}abling. Il tabling è utilizzato per memorizzare il risultato di un calcolo in memoria, permettendo quindi una rapida consultazione del risultato al posto del ricalcolo. Per questo motivo, il modulo \verb|planner| di Picat ne fa largo uso. 
\end{itemize}

Inoltre, Picat supporta le maggiori caratteristiche dei linguaggi logici, obbligando però il programmatore ad usarle in modo esplicito, come per il caso dell'unificazione, del non determinismo, del tabling e dei vincoli.

\section{Il sistema Picat}

Picat è un linguaggio interpretato. I programmi Picat però, prima di essere eseguiti, sono compilati. Il compilatore non trasforma i sorgenti in codice macchina, ma depura il programma da tutto quello che non è necessario per l'esecuzione, esegue alcuni controlli statici e riscrive alcuni costrutti sintatticamente (e.g. i cicli).

Tramite l'interprete è quindi possibile caricare, compilare ed eseguire programmi Picat, possibilmente anche modalità debug.

\paragraph{Esempio}
\begin{verbatim}
> picat
Picat> compile(program)
Picat> load(program)
Picat> main.
\end{verbatim}



\section{Tipi di dato}

Picat è un linguaggio tipato dinamicamente: il controllo dei tipi avviene a runtime. Le variabili in Picat possono avere o non avere un valore. Una variabile è libera fintanto che non viene vincolata ad un valore (operazione di bounding), inoltre possono avere degli attributi rappresentati come mappa chiave-valore.

I tipi primitivi di Picat sono il numero intero, reale e l'atomo. Un atomo è un nomo iniziante con lettera miniscola o una stringa quotata da apici singoli.

I tipi composti invece sono le liste \verb|[|$t_1, \ldots, t_n$\verb|]| e le strutture \verb|$s(|$t_1, \ldots, t_n$\verb|)|, dove $s$ è il nome della struct, ed $n$ è l'arità. Il dollaro viene impiegato per distinguere una chiamata di funzione da una struct. Picat inoltre mette a disposizione tipi composti speciali come \emph{strings}, \emph{arrays}, \emph{maps}, \emph{sets} and \emph{heaps}.

\paragraph{Esempio}
\begin{verbatim}
Picat> V1 = X1, V2 = _ab, V3 = _       % variabili

Picat> N1 = 12, N2 = 0xf3, N3 = 1.0e8  % numeri

Picat> A1 = x1, A2 = ’_AB’, A3 = ''    % atomi

Picat> L = [a,b,c,d]                   % liste

Picat> write("hello"++"picat")
[h,e,l,l,o,p,i,c,a,t]                  % stringhe

Picat> print("hello"++"picat")
hellopicat

Picat> writef("%s","hello"++"picat")
hellopicat                             % write con formato

Picat> writef("%-5d %5.2f",2,2.0)
2
2.00                                   % write con formato

Picat> S = $point(1.0,2.0)             % struttura

Picat> S = new_struct(point,3)
S = point(_3b0,_3b4,_3b8)              % creazione di una struttura

Picat> A = {a,b,c,d}                   % array

Picat> A = new_array(3)
A = {_3b0,_3b4,_3b8}                   % creazione di un array

Picat> M = new_map([one=1,two=2])
M = (map)[two = 2,one = 1]             % creazione di una map

Picat> M = new_set([one,two,three])
M = (map)[two,one,three]               % creazione di un insieme

Picat> X = 1..2..10
X = [1,3,5,7,9]                        % ranges

Picat> X = 1..5
X = [1,2,3,4,5]
\end{verbatim}

Picat possiede una ricca libreria di funzioni di utilità sui tipi di dato che permettono, tra le altre cose, di verificare qual è il tipo di una variabile, convertire tipi ad altri tipi, aggiungere elementi ad un insieme, eccetera.

\section{Predicati}

I predicati sono una componente fondamentale di Picat. Un predicato ha esito positivo o negativo, a meno di eccezioni. Un predicato può esplorare tutte le soluzioni tramite il backtracking implementato in Picat.

In Picat i predicati sono definiti con regole di pattern-matching. Le regole utilizzabili per la definizione di un predicato sono $Head, Cond => Body$ e $Head, Cond\ ?=> Body$ che rappresentano rispettivamente la regola "non-backtrackable" e "backtrackable". La testa $Head$ del predicato è $p(t_1, \ldots, t_n)$ con $p$ il nome del predicato e $n$ l'arità. Le condizioni $Cond$ sono goal opzionali che specificano l'applicabilità della regola. $Body$ è li corpo del predicato.
Quando una chiamata $C$ è applicabile, Picat riscrive $C$ in $Body$. Se la regola "non-backtrackable" è usata allora la riscrittura è permanente, altrimenti il programma farà backtracking su $C$ se $Body$ fallisce e verrà provata la prossima regola.

\paragraph{Esempio (non backtrackable)}
\begin{verbatim}
membchk(X,[X|_]) => true.
membchk(X,[_|L]) => membchk(X,L)
\end{verbatim}
il cui output è
\begin{verbatim}
Picat> membchk(X,[1,2,3]).
no
\end{verbatim}

\paragraph{Esempio (backtrackable)}
\begin{verbatim}
member(X,[Y|_]) ?=> X=Y.
member(X,[_|L]) => member(X,L).
\end{verbatim}
che può essere usata per risultati come
\begin{verbatim}
Picat> member(X,[1,2,3])
X=1;
X=2;
X=3;
no
\end{verbatim}

\section{Funzioni}

Picat permette di definire delle funzioni, ovvero oggetti che hanno sempre esito positivo e che ritornano un solo valore. La sintassi di una funzione è $Head, Cond = Result => Body$. Le funzioni in picat sono uno zucchero sintattico per un predicato cos' formato: $HeadResult, Cond => Body$ dove $HeadResult = p(t_1, \ldots, t_n, r)$.

\paragraph{Esempio}
\begin{verbatim}
fib(0)=F => F=1.
fib(1)=F => F=1.
fib(N)=F,N>1 => F=fib(N-1)+fib(N-2).
\end{verbatim}

\section{Assegnamenti e Cicli}

Picat cerca di agevolare notevolmente i programmatori affini a linguaggi imperativi. Per questo motivo offre funzionalità come l'assegnamento, tipicamente non disponibile in un linguaggio logico,
e cicli, che nei linguaggi logici si realizzano con la ricorsione.

\subsection{Assegnamento}

Anche l'assegnamento, come altri comandi, è uno zucchero sintattico. Un assegnamento in Picat $X := Y$ viene gestito a tempo di compilazione creando nuovi nomi di variabile. Le nuove variabili conterrano il nuovo valore dopo l'assegnamento e verranno usati nella porzione di programma successiva all'asegnamento al posto della vecchia variabile.

\paragraph{Esempio}
\begin{verbatim}
test => X=0, X:=X+1, X:=X+1, write(X).
\end{verbatim}
ha come output $2$, e con la compilazione viene riscritto in
\begin{verbatim}
test => X=0, X1=X+1, X2=X1+1, write(X2).
\end{verbatim}
avente anch'esso output $2$.

\subsection{Cicli}

Picat supporta tre tipo di cicli: \verb|foreach|, \verb|while| e \verb|do-while|. Il \verb|foreach| viene usato per implementare un particolare tipo di ciclo su liste denominato \emph{list comprehension}.

\paragraph{Esempi}\
\begin{lstlisting}
% esempio di foreach
L = [2, 3, 5, 10],
foreach(I in L, J in 1 .. 10, J mod I != 0)
	printf("%d is not a multiple of %d%n", J, I)
end.

% esempio di while
I = 1,
while (I <= 9)
	println(I),
	I := I + 2
end.

% esempio di do while
J = 6,
do
	println(J),
	J := J + 1
while (J <= 5).
\end{lstlisting}

Mentre per i tipi di cicli più comuni la semantica è piuttosto ovvia, per le list comprehension la cosa è più interessante. Le list comprehension sono funzioni che servono a costruire una lista iterando valori di altre liste.

La list comprehension ha la seguente seguente sintassi
\begin{lstlisting}
[T : $E_1$ in $D_1$, $Cond_1$ , $\ldots$, $E_n$ in $D_n$ , $Cond_n$]
\end{lstlisting}

\paragraph{Esempio}\
\begin{lstlisting}
picat> L = [(A, I) : A in [a, b], I in 1 .. 2].
L = [(a , 1),(a , 2),(b , 1),(b , 2)]
\end{lstlisting}

\section{Funzionalità avanzate}

In questa sezione si da una panoramica delle funzionalità "avanzate" in Picat. Anche in questo caso si raccomanda al lettore l'approfondimento dei dettagli in \cite{PicatGuide}.

\subsection{Tabling}

Il tabling è un meccanismo di caching che permette di memorizzare la chiamata ad un predicato e relativo risultato per evitare cicli infiniti o per ridurre il costo computazione dovuto alla grossa ridondanza dei calcoli.

\paragraph{Esempio}\
\begin{lstlisting}
table
fib(0) = 1.
fib(1) = 1.
fib(N) = fib(N-1)+fib(N-2).
\end{lstlisting}
Senza tabling \verb|fib(N)| ha costo esponenziale in \verb|N|, con il tabling invece il costo è lineare.

Il tabling è abilitato usando la keyword \verb|table| prima della definizione del predicato.

\subsection{Moduli}

I moduli sono una funzionalità molto utile di Picat e permettono al programmatore un'organizzazione comoda del codice sorgente, suddivisa su più file (moduli, appunto).

Un modulo è un file sorgente con estensione \verb|.pi|. Un modulo inizia con la dichiarazione di se stesso nella forma \verb|module| $Name$, dove $Name$ corrisponde al nome del file.

Picat fornisce un sistema molto semplice di import, consentendo così di accedere a funzionalità di altri file sorgente. La sintassi per l'import è 
\begin{lstlisting}
import $Name_1, \ldots, Name_n$.
\end{lstlisting}

E' possibile chiamare predicati di un modulo specificando completamente il nome del predicato con \verb|m.p()|, dove \verb|m| è il nome del modulo e \verb|p| è il nome del predicato. E' consentito proibire l'utilizzo di un predicato da moduli esterni appondendo la keyword \verb|private| prima della definizione del predicato.

\subsection{Eccezioni}

Picat fornisce un semplice sistema di gestione delle eccezioni, con cui è possibile gestire eventi eccezionali ed errori non previsti.

La sintassi per il lancio delle eccezioni è 
\begin{lstlisting}
throw Exception
\end{lstlisting}
mentre la cattura e la gestione dell'eccezione si effettua tramite
\begin{lstlisting}
catch(Goal, Exception, Handler)
\end{lstlisting}


\printbibliography

\end{document}