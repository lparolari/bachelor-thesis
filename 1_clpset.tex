% subdoc.begin.tex

\ifdefined\COMPLETE
\else
%\documentclass[11pt]{article}
\documentclass[12pt,a4paper,openright]{book} % oneside

\usepackage{lmodern}         % font package.
\usepackage[T1]{fontenc}     % define T1 charset for out files.
\usepackage[italian]{babel}  % italian latex typo conventions.
\usepackage[utf8]{inputenc}  % italian symbols.
\usepackage{amsmath}         % math features.
\usepackage{amsthm}          % math theorems.
\usepackage{amssymb}         % math symbols.
\usepackage{listings}        % embed programming language in latex.
\usepackage{stmaryrd}        % symbols for theoretical computer science.
\usepackage{hhline}          % better horizontal lines in tabulars and arrays.
%%\usepackage{vmargin}         % various page dimensions.
\usepackage{hyperref}        % hypertext support.
\usepackage{makeidx}         % for creating indexes.
\usepackage{nicefrac}        % inline fractions.
\usepackage{marginnote}      % notes in the margin, even where \marginpar fails.
\usepackage{xr}              % references to other latex documents.
\usepackage{subfiles}        % multifile support.
\usepackage{geometry}        % interface for document dimension.
\usepackage{graphicx}        % enhanced support for graphics.
\usepackage{fancyhdr}        % extensive control of page headers and footers.
\usepackage{lipsum}          % generate dummy text.
\usepackage[backend=biber,style=numeric]{biblatex}  % bib management. %bibtex


%\usepackage{showframe}      % shows page frames.


%   CONFIGS
% ***********************************************************
\hypersetup{colorlinks=true, urlcolor=blue, linkcolor=blue}
\dominitoc

%   RESOURCES
% ***********************************************************
\subfile{prooftree}
\input{macros.tex}
\addbibresource{biblio.bib}



\begin{document}
\fi


% ************************
%%% begin chapter

%\chapter{CLP(SET)}
\chapter{Linguaggi a vincoli basati su insiemi e relazioni binarie}

In questo capitolo vengono presentati ed inquadrati nel loro contesto i linguaggi $\clpset$ e $\lbr$.

$\clpset$ è un linguaggio a vincoli, notevolmente espressivo, basato su insiemi; $\lbr$ è invece un linguaggio che estende ed include $\clpset$, interamente basato su vincoli su relazioni binarie.





\section{CSP}
Un problema di soddisfacimento di vincoli è composto da tre componenti:
\begin{itemize}
\item un insieme $X$ di variabili ${X_1, \ldots, X_n}$;
\item un insieme $D$ di domini ${D_1, \ldots, D_n}$, uno per variabile;
\item un insieme $C$ di vincoli che specificano le combinazioni di valori possibili per le variabili.
\end{itemize}

Ogni dominio $D_i$ specifica un insieme di valori possibili per la variabile $X_i$.


\section{$\clpset$}
La nozione di \emph{set} è una componente comune nella progettazione di programmi, ma sono pochi i linguaggi che forniscono gli insiemi come struttura dati elementare.
In [2000-CLP(SET)] vengono menzionate alcune eccezioni di linguaggi che si basano su insiemi come SETL, B ed il linguaggio Z, utilizzato per descrivere la specifica formale di programmi. Qualche eccezione esiste anche nel campo dei \emph{database deduttivi} e, più recentemente, anche come general purpose programming language. Ad ogni modo, questi linguaggi impongono dei limiti sul tipo di insiemi espressibili o sulle capacità computazionali degli stessi. Ad esempio, in molti ambiti, si richiede che gli insiemi siano totalmente specificati: non sono ammesse \emph{variabili} libere.


 Può essere incluso nella famiglia CLP (Constraint Logic Programming), che a sua volta può essere inquadrata nella categoria dei CSP (Constraint Satisfaction Problem), ovvero problemi di soddisfacimento di vincoli. 

Il linguaggio $\clpset$ è stato successivamente esteso ed incluso nel linguaggio $\lbr$, basato su relazioni binarie.

% ************************
%%% end chapter

% subdoc.end.tex

\ifdefined\COMPLETE
\else
\end{document}
\fi
